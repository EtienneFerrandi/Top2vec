\documentclass{article}
\usepackage{fontspec}

\usepackage[xetex]{hyperref}

\usepackage{inputenc}
\usepackage[T1]{fontenc}
\usepackage[english,french]{babel}
\usepackage[margin=2.5cm]{geometry}
\usepackage{setspace}
\onehalfspacing
\setlength\parindent{1cm}
\usepackage{lettrine}
\usepackage{xspace}
\usepackage{pifont}
\usepackage{tocbibind}

\usepackage{csquotes}
\usepackage[defernumbers=true, backend=biber, sorting=nyt, style=enc,citestyle=verbose,maxbibnames=10]{biblatex}
\bibliography{}
\addbibresource{}
\nocite{*}

\usepackage{url}

\pagenumbering{arabic}
\setcounter{page}{1}

\usepackage{fancyhdr}
\pagestyle{fancy}

\usepackage{graphicx}
\graphicspath{{./images/}}
\usepackage{float}

\usepackage{multirow}
\usepackage{array}
\usepackage{longtable}
\usepackage{rotating}
\usepackage{caption}

\usepackage{enumitem}

\usepackage{verbatim}
\usepackage{verbatimbox}

\title{Rendu pour le séminaire TAL. Topic Modelling sur les textes du corpus Savoirs}
\author{Etienne Ferrandi}
\date{May 2022}

\begin{document}

\maketitle

Le corpus Savoirs (\url{https://savoirs.app/}) est une bibliothèque numérique en libre accès répertoriant des articles d'anthropologie des savoirs de l'Antiquité à nos jours. 

Notre participation au défi 2 du hackathon Savoirs qui s'est tenu les 12-13 mai 2022 au Campus Condorcet, "analyse et visualisation du corpus Savoirs" nous a conduit à réaliser une analyse de modélisation à sujet (\textit{topic modelling}). 

Les données ont été récupérés grâce à un \textit{parser} sur les fichiers XML/TEI du corpus disponibles à l'adresse suivante (\url{https://github.com/PSIG-EHESS/HackathonSavoirs/tree/main/CorpusTEI}). A chaque ligne de la colonne "Texte" du tableau généré suite à ce \textit{parsing}, figure un texte du corpus. Sont incluses les notes de bas de page.

Au cours d'une phase de \textit{pre-processing}, ont été enlevés les \textit{stopwords} français, anglais et italiens avec \textit{Spacy} car il s'agit des trois langues principales du corpus, ainsi que la ponctuation. La lemmatisation a été réalisée grâce à \textit{nltk}. Pour réduire la durée de l'apprentissage machine, nous avons découpé chaque texte en phrases. 

Nous avons chargé un modèle pré-entraîné multilingue, "distiluse-base-multilingual-cased", téléchargé grâce à une dépendance du package \textit{top2vec}, \textit{sentence-transformers}, afin de gérer le multilinguisme du corpus (principalement français et anglais mais aussi italien). Ainsi, on peut avoir dans un même \textit{topic} des mots qui sont des traductions dans une autre langue.

Nous avons ainsi entraîné un modèle \textit{Top2vec} sur les phrases des textes du corpus. Il est ressorti un total de 973 \textit{topics} dont un bon nombre sont pertinents. Certains sont des listes d'abréviations ou de mots sans lien évident entre eux, comme le \textit{topic} 16 et peuvent donc ne pas être retenus pour l'analyse.

Par exemple, le \textit{topic} 1 a trait à l'Antiquité gréco-romaine. Il est caractérisé par la présence entre-autres des mots "grec", "mythe", "théologie", "archéologie", "philologie", "philosophie", "latin" et leurs dérivés. Son nuage de mot fait apparaître une prédominance du mot "grec" et de ses dérivés.

\begin{figure}[!h]
    \centering
    \includegraphics[width=16cm]{topic 1.png}
    \label{1}
\end{figure}
\newpage

Sur la thématique grecque, on identifié d'autres \textit{topics} qui s'y rapportent. Avec \verb+search_topics+, on est en mesure d'identifier les \textit{topics} relatifs à un mot-clé donné, ici "grec". Chaque \textit{topic} est listé dans l'ordre décroissant de leur score de similarité cosinus avec le mot-clé. L'un d'entre eux est le \textit{topic} 47 dont le centroïde affiche un score de similarité cosinus avec le mot 'grec' de 0.617\footnote{Pour le \textit{topic} 1, il est de 0.642}. Son nuage de mots ne paraît pas indiqué de thème dominant mais on note la présence de quelques vocables grecs, "\textit{enargeia}", "\textit{euripide}", "\textit{theodore}", "\textit{isocrate}", "\textit{ethos}", "\textit{ekphrasis}", "\textit{eusèbe}".

\begin{figure}[!h]
    \centering
    \includegraphics[width=16cm]{topic 47.png}
    \label{47}
\end{figure}

On peut identifier les textes qui se rapportent à ces \textit{topics}, d'abord en affichant les phrases ayant avec le centroïde du \textit{cluster} du \textit{topic} sélectionné le score de similarité cosinus le plus élevé grâce à la commande \verb+search_documents_by_topic+, ensuite en recherchant son occurrence dans le tableau contenant les textes, leur titre et leur auteur, la date et le nom de la revue de publication.

Ainsi, pour le \textit{topic} 47, les phrases retenues sont principalement des noms propres grecs ou latins, "\textit{Euthyd}", "\textit{Themis}", "\textit{Isocrate}", "\textit{Herrenium}", "\textit{Thucydide}", "\textit{Andriotis}", "\textit{Xénophane}".

Pour chaque \textit{topics} 1 puis 47, les phrases ayant le score de similarité cosinus le plus élevé avec le centroïde du \textit{cluster} ont été retrouvées dans le tableau par une simple recherche. On a ainsi pu identifier une liste non-exhaustive de textes qui se rapportent à ces \textit{topics} (cf. annexe).


En guise de conclusion, quelques perspectives d'amélioration : supprimer les références dans les notes de bas de page qui contiennent beaucoup de mots-clés susceptibles d'être survalorisés dans le calcul du \textit{topic modelling} ; mettre de côté les entités nommées homonymes de noms communs (ex : "Professeur Lesson").


\include{main/appendix}

\appendix
\part*{Annexes}
\addcontentsline{toc}{part}{Annexe}
\pagestyle{myheadings}
\markboth{Annexe}{Annexe}

\begin{itemize}
    \item Renée Koch-Piettre, " Inscrire un serment en Grèce ancienne : couper et verser ", Cahiers " Mondes anciens " [En ligne], 1 | 2010
    \item Aurélien Berra. Manier le thésaurus grec. Christian Jacob. Lieux de savoir. II. Les mains de l'intellect, Albin Michel, pp.555-578, 2011
    \item ZANGARA, Adriana. Mettre en images le passé : L’ambiguïté et l’efficacité de l’enargeia dans le récit historique In : Dossier : Phantasia [en ligne]. Paris-Athènes : Éditions de l’École des hautes études en sciences sociales, 2004
    \item Karel Thein. La force du vraisemblable. Philostrate, Callistrate et l’image à l’épreuve de l’ekphrasis. Mètis - Anthropologie des mondes grecs anciens, Daedalus/EHESS, 2008, N.S.6, pp.315-344
    \item Carlos Lévy. De la Grèce à Rome : l'espace-temps des philosophes antiques. Lieux de savoir, 1. Espaces et communautés, Albin Michel, 2007, p. 1019-1048
    \item Naïs Virenque, " De l’arbre à l’arborescence. Mémoire et matérialité des raisonnements du Moyen Âge à l’époque moderne ", in Matières à raisonner, textes réunis par Françoise Briegel et Jean-François Bert, en ligne, Savoirs. Le fil des idées, 2022
    \item CORRADI, Michele. Protagoras dans son contexte. L'homme mesure et la tradition archaïque de l'incipit In : Dossier : Tekhnai/artes [en ligne]. Paris-Athènes : Éditions de l’École des hautes études en sciences sociales, 2007
    \item Pierre Chiron. Les Arts rhétoriques gréco-latins : structures et fonctions. Mètis - Anthropologie des mondes grecs anciens, Daedalus/EHESS, 2006, Tekhnai / artes. Entre la théorie et la pratique, le savoir-faire., pp.101-134
    \item Vincent Azoulay et Aurélie Damet, " Paroles menaçantes et mots interdits en Grèce ancienne : approches anthropologiques et juridiques ", Cahiers " Mondes anciens " [En ligne], 5 | 2014
    \item Marie Saint Martin, " L’invention de la tragédie selon Pierre Brumoy : de quelques pièges du relativisme ", Cahiers " Mondes anciens " [En ligne], 11 | 2018
    \item DELATTRE, Charles. ΑΙΤΙΟΛΟΓΙΑ : mythe et procédure étiologique In : Dossier : Images mises en forme [en ligne]. Paris : Éditions de l’École des hautes études en sciences sociales, 2009
    \item Alessandro Buccheri, " Deiknūnai tēn phūsin : valider l’identité filiale dans le Philoctète de Sophocle ", Cahiers " Mondes anciens " [En ligne], 8 | 2016
    \item Michel Briand, Florence Dupont et Vivien Longhi, " La " civilisation " : critiques épistémologique et historique ", Cahiers " Mondes anciens " [En ligne], 11 | 2018
    \item MURRAY, Penelope. Qu’est-ce qu’une Muse ? In : Dossier : S'habiller, se déshabiller dans les mondes anciens [en ligne]. Paris : Éditions de l’École des hautes études en sciences sociales, 2008
    \item Anastasia Serghidou. Logiques des émotions, sociabilité et construction de soi : De la caractérologie aristotélicienne au prosopôn tragique. Mètis - Anthropologie des mondes grecs anciens, Daedalus/EHESS, 2011, Dossier : Émotions, N.S.9, pp.81-100
    \item Adriana Zangara, " Voir l’histoire. Théories anciennes du récit historique. Présentation ", Anabases, 7 | 2008, 249-256
    \item Nicolas Davieau, " Montrer le corps : prouver le philosophe ? Le corps des philosophes dans la statuaire antique ", Cahiers " Mondes anciens " [En ligne], 8 | 2016
    \item Isabelle Tassignon, " Lawrence Durrell, les ruines et l’histoire ", Anabases, 18 | 2013, 197-212
    \item Frédéric Colin, " Comment la création d’une " bibliothèque de papyrus " à Strasbourg compensa la perte des manuscrits précieux brûlés dans le siège de 1870 ", La Revue de la BNU, 2 | 2010, 24-47
    \item Luciano Canfora, " La culture classique à Rovereto dans la première moitié du 18e siècle : parcours de lecture de G. Tartarotti ", La Revue de la BNU, 6 | 2012, 90-101
    \item Madalina Dana, " Centre et périphérie : la mobilité culturelle entre la mer Noire et le monde méditerranéen dans l’Antiquité ", dans Chr. Jacob (éd.), Lieux de savoir. Espaces et communautés, Paris, Albin Michel, 2007, p. 924-941
    \item Pierre Chiron. "Le travail de la mémoire, Lieux de Savoir. Espaces et communautés", Paris, Albin Michel, 2007, p. 675-680
    \item Detienne, Marcel. " De l’efficacité en raison pratique. Approches comparatives ", Po\&sie, vol. 137-138, no. 3-4, 2011, pp. 231-241
    \item Leopoldo Iribarren. De la pertinence d’un motif idéaliste : Sur les présocratiques d'André Laks. Mètis - Anthropologie des mondes grecs anciens, Daedalus/EHESS, 2008, N.S.6, pp.299-313
    \item Georgoudi, Stella. “Comment régler les theia pragmata : Pour une étude de ce qu’on appelle " lois sacrées "”. Georgoudi, Stella. Dossier : Normativité. Paris-Athènes : Éditions de l’École des hautes études en sciences sociales, 2010. (pp. 39-54)
    \item Corinne Bonnet. Errata, absurditates, deliria et hallucinationes. Le cheminement de la critique historique face à la mythologie phénicienne de Philon de Byblos : un cas problématique et exemplaire de testis unus. Anabases - Traditions et réceptions de l’Antiquité, E.R.A.S.M.E., 2010, 11, pp.123 - 136
    \item Anthony Andurand, " Fabrique du mythe et production des savoirs : la Grèce des hellénistes allemands au miroir du Griechenmythos (1790-1945) ", Anabases, 15 | 2012, 230-237
    \item SKARSOULI, Pénélope. Des papyri grecs en contexte funéraire : L’exemple de " L’Empédocle de Strasbourg " In : Dossier : Normativité [en ligne]. Paris-Athènes : Éditions de l’École des hautes études en sciences sociales, 2010
    \item Manfrini, Ivonne, et Nina Strawczynski. “Une forge ambiguë”. Manfrini, Ivonne, et Nina Strawczynski. Dossier : Tekhnai/artes. Paris-Athènes : Éditions de l’École des hautes études en sciences sociales, 2007. (pp. 51-90)
\end{itemize}

\end{document}

